% CREATED BY MAGNUS GUSTAVER, 2020
\chapter{Methods}

The methods chapter describes the methods which have been used in an attempt to reach the end goal of the project. In addition to this, it is explained why these specific methods were chosen. Any additional information which might be needed to understand the used methods and reasoning behind them, is located within the previously presented theory section.

\section{Configuring the toolchain}

The tools and software used to program and control the Autobike had already been decided to a large extent by the previous groups which worked on the project, as well as other stakeholders. It was therefore necessary to install these tools before the previous work could be examined to the fullest extent, and before further work could commence.

The installation process of these tools, will be presented in the upcoming subsections. Installation steps of other pieces of software will also be presented, together with their purpose and the reasons behind why they were decided to be used.

\subsection{LabVIEW}

The parts of the Autobike software that were already written had been done in LabVIEW. The LabVIEW program itself was therefore installed with the purpose of being able to view this code. Since a MyRIO-1900 had been selected to run the code, a special toolchain of plugins and programs which supports the configuration the device, as well as uploading and running code on it, also had to be installed. These programs were all first party tools available on the National Instruments website\footnote{All the required software for developing for the MyRIO is available in the \textit{LabVIEW myRIO Software Bundle} available for download on the National Instruments website: \raggedright\url{https://www.ni.com/sv-se/support/downloads/software-products/download.labview-myrio-software-bundle.html}}, or in the package manager which was installed alongside LabVIEW. 

It should be noted that the installation of LabVIEW and the necessary packages is more time consuming and problem-ridden than what might first be expected. The installation itself is not uncommon to take multiple hours, especially if the computer's storage space runs out in the middle of an installation. This step was further delayed due to downloads from the National Instruments website not working for several days.

\subsection{Text editor and build tools for C code}

Some of the more advanced algorithms used in the Autobike were better suited to be written using C code, and called by the main LabVIEW program. These algorithms include code written for another version of the bike with the purpose of controlling the steering motor, as well as code needed in the communication with the forward motor.

To be able to use C code with LabVIEW, a toolchain capable of both writing C code, but also (cross) compiling it, was required. The toolchain would also need to create shared libraries as well as uploading these to the target. The selected software, and the rationale behind the choices are further discussed below.

\subsubsection{Text editor / IDE: Visual Studio Code}

National Instruments has published a guide for setting up a C development toolchain using the IDE (integrated development environment) Eclipse \cite{NationalInstruments2021GettingEdition}. However due to preferential reasons and the project members' familiarity with Visual Studio Code (VS Code), it was chosen as the editor instead of Eclipse. 

The basics of the previously mentioned guide can still be followed even though the details differ. There is a guide published on the National Instruments forum that describes how VS Code can be set up to work with the MyRIO as well \cite{J2020NICode}. Although this is not an official guide, it is authored by a NI employee. After installing VS Code, the \textit{C/C++ for Visual Studio Code} extension is recommended as it allows for code-completion aids and debugging capabilities for code written in C \cite{VanLiew2022C++Code}. 

\subsubsection{Compiler and Build Tools: GCC, CMake and Ninja}

Before the C code can be called from within LabVIEW when it is running on a MyRIO, it must first be compiled and built into a shared library as mentioned in section \ref{theory:text_based}. The previously mentioned guide for setting up VS Code to develop for a MyRIO also goes through how to set up the needed build tools \cite{J2020NICode}. 

The compiler used in the guide is a version of GCC made to run on Windows, and compile the code for Linux running on ARMv7; this compiler is used since VS Code is running on a computer using windows and the myRIO has an ARM processor. To aid the build process CMake, which is "a family of tools designed to build, test and package software", is used to generate makefiles \cite{Kitware2022CMake}. These files can then be used by Ninja, a build system, to run the compiler and create a shared library or \texttt{.so} file \cite{Hasse2022NinjaSpeed}.

\subsubsection{FTP Client: Filezilla}

When the shared library or \texttt{.so} file has been created, this file needs to be transferred to the myRIO. To accomplish this, the guide recommends Filezilla, which is a "cross-platform FTP, FTPS and SFTP client" \cite{FileZilla2022ClientFeatures}. Filezilla was used to connect to the myRIO and upload the files as described in \ref{theory:text_based}.

\section{Examining Previous Work and the Current State of the Project}

When the tools necessary for viewing and running LabVIEW code had been set up, the existing code was examined and tested. This was done in order to learn what parts of the Autobike's software had not yet been completed. It was concluded that all hardware except the forward motor could be interacted with in some way from the main LabVIEW program. 

All sensors, including the GPS, gyro, accelerometer, hall sensor (for measuring the forward speed), and position sensor (to measure the steering wheel position) was visually represented by graphs or similar on the LabVIEW front panel. The steering motor duty cycle could be controlled using a slider, and the emergency stop could stop the program except for in some edge cases (it should however be noted that the emergency stop always physically broke the circuit).

In addition to the main LabVIEW program, several other VIs were present in the folder handed over by the previous project group. Some of these were used as sub-VIs in the main program, but the majority of them seemed to be test code that had been used during the development process. One of these programs appeared to be an attempt at communicating with the forward motor; attempts to make the code to work were however unsuccessful, as further described in \ref{method:forward}.

\section{Steering Motor}

At the start of this project progress related to being able to control the steering motor had already been made by the previous project group. Excluding the hardware, LabVIEW code which could communicate with the steering motor controller, using a PWM signal, had already been completed. The duty cycle of the PWM signal had a value ranging from 0\% to 100\%, where a values above 50\% would rotate the motor clockwise, and values below 50\% rotate the motor anti-clockwise, a value of 50\% would stop the motor. 

The code consisted of three states: a start state, the control loop, and the end state. The start state made sure the motor was still and thereafter disabled the pin where the signal was sent, and waited until the user told the motor to start with a "Go" button. After being told to start, the control loop would be entered and the pin was enabled, which allowed the user to control the speed and direction of the motor using a slider; the slider changed the PWM signal sent to the motor controller. Finally the end state would be entered if a stop signal was given, for example by activating the physical emergency stop button, this state also stops the motor and disables the pin.

Testing the code by starting the program, pressing the "Go" button, and changing the PWM signal, the code proved to be working as intended; the motor could rotate in both directions and change speed accordingly. Therefore it was decided the code was going to be used, with minor alterations in the form of exchanging the slider for the previously mentioned control algorithm.

The following subsections describes the methodology and process of completing the software for the steering motor. This includes identifying and translating relevant code from another version of the Autobike, using the gyroscope built into the IMU together with the algorithm to control the steering motor, and implementing safety limits to the motor's range of motion.

\subsection{Transferring Control Algorithm From the Black Bike}

To reach the end goal of creating a bike which can balance on its own, code that was written for the other version of the bike, the so called black bike, were to be converted as to work with the current version which this project relates to. The software for the black bike had been written completely in Python, however it still used a MyRIO. 

Before the code could be used, the parts of the code which are relevant to the current bike, had to be identified. This process, followed by the conversion of the code to C, and its integration with LabVIEW is described below.

\subsubsection{Identifying Relevant Code}

Since the software used on the black bike was written entirely in Python, a large amount of the code had the same or similar functionality as the LabVIEW code for the red bike. For reference, the folder that was received contained in total 156 python files with a combined 20 253 lines of code\footnote{As reported by running the following commands in a Linux terminal:\\\texttt{find . -mindepth 1 -type f -name "*.py" -printf x | wc -c} and\\\texttt{wc -l `find . -type f -name "*.py"`}}. Code which was used for loading simulations from CSV files were also present in most of files. From this it could be concluded that almost all of the received code was not relevant for the red bike and would not have to be used.

Trajectory tracing is also not a part of this project's goal, meaning the only part of the code that would be needed was the code which relates to keeping the bike stable. This code was located in the file \texttt{controller.py} and more specifically in the function \texttt{keep\_the\_bike\_stable}. The function also called another function, \texttt{update}, which implemented a PID controller in parallel form according to:

\begin{equation*}
    u(t) = K_p e(t) + K_i \int_{0}^{t} e(t)dt + K_d \frac{de}{dt}
\end{equation*}

The value returned from this function is then further sent to another function, named \texttt{controller\_set\_handlebar\_angular\_velocity}. This final function calculates the angular velocity of the motor based on some safety limits, and calls a function to convert the angular velocity to a duty cycle. In Python, this conversion is done using the \texttt{interp} function from the \textit{NumPy} library with the velocity on the x-axis, and PWM on the y-axis. In C this as been converted to \texttt{50 + rpm * 40.0 / 1000.0}, which gives a value between 0 and 100 when \texttt{rpm} is in the range of the motors angular velocity.

\subsubsection{Converting Python to C}

The previous project group that worked on the bike considered using a language called Cython to compile the Python code to C, and thus avoid having to manually convert the code \cite{AronssonKarlsson2022PROJECTAUTOBIKE}. The group mentions that the translation created code which was confusing and hard to understand, and is was concluded that "Cython is not useful for this project". It became further apparent that Cython was not needed after the relevant Python code had been identified. This conclusion was made based on the relatively small amount of code that would actually have to be converted. It was therefore decided that the code was going to be converted manually.

The conversion process started with a new C project being created and the identified Python functions being copied there. The code was then adapted to fit the C syntax, at the same time as irrelevant code was removed. This irrelevant code mainly included lines that was only called when a simulation file were present, but some code was also removed since it did not have any effect on the rest of the program (for example variables that were written to, but then never used). The safety limits were also removed and not implemented in the C code because it was decided they would be created in LabVIEW instead. This decision was based upon two reasons: to facilitate a better understanding and easier debugging of the code by having as much code written in LabVIEW as possible, as well as being able to use LabVIEW to log when a safety limit had been exceeded.

Another change that was made to the control algorithm when converting it into C code was the removal of code that depended on the time, i.e. the integral and derivative action in the PID controller. The integral of the error was instead implemented by adding the current error to the previous errors once every time the function was called. The derivative of the error was calculated as the difference between the current and previous error (the error the last time the function was called).

\subsection{Automating steering motor control}

To automate the control of the steering motor, meaning automatically calculating the duty cycle deciding which direction and how quickly the motor should rotate, a shared library file of the converted C code was created and uploaded to the MyRIO. The bike's roll rate\footnote{Roll rate in this case means the speed of which the bike rotates along its lengthwise axis.}, read from the IMU's gyroscope, as well as the three respective PID gains were then connected to a \textit{Call Library Function Node} in LabVIEW. The node was set to call one of the functions from the C algorithm. The called function uses these four parameters and returns the calculated duty cycle.

The received duty cycle can theoretically have a value outside the range of 10\% to 90\%. Because of this the duty cycle is coerced to fall within this range using the \textit{In Range and Coerce} function in LabVIEW.

For safety reasons the range which the wheel can turn within needs to be limited. By using the steering motor encoder, the position of the steering motor can be calculated in degrees from when the wheel is straight. As mentioned in \ref{theory:encoder} the encoder has a resolution of 500 counts per revolution, but when turning the wheel a full rotation, the resulting counts are approximately 220 000; the reason for this is unknown. Thus the resulting factor for converting between counts and degrees is $360 / 220000$. In LabVIEW the rotational position is then compared with a limitation, configured from the main program's front panel. The current limit is set to 45 degrees in each direction, as this is the limit used for the black bike. If the limit is exceeded the duty cycle is set to 50\%, meaning the motor should stop.
 
The duty cycle is then sent to a node in LabVIEW which creates a PWM signal. This signal replaces the previously existing slider and is in turn sent to the hardware pin (pin 19) which the steering motor controller is connected to.

\subsubsection{Calibrating Gyroscope}

When the automated control of the steering motor was tested, it was discovered that the motor rotated even if the bike was not moving. During an investigation of the graphs that plotted the gyroscopes values, it appeared as though the average value of the rotational rate around the roll axis was non-zero, even when the bike was stationary. This seems to be a case of a "constant bias error" \cite{Beavers2017TheGyro}.

To counteract the error, the bike was kept stationary over a time while the values returned from the gyroscope for the roll axis was recorded. The average of these recordings was then calculated and the negative of this average was added to the gyroscope's output to correct the error. It should be noted that the error or "constant bias error" might drift over time, meaning the error might change. If this occurs the average offset could be calculated again, or more sophisticated methods could be utilized as mentioned in \cite{Beavers2017TheGyro}.

\section{Forward Motor} \label{method:forward}

When the project was received, the main LabVIEW program contained no code with the purpose of controlling the forward motor. The reason for this code not being created was that "the command used for controlling the forward motor controller were not found" as described in the previous project's report \cite{AronssonKarlsson2022PROJECTAUTOBIKE}.

The MyRIO communicates using UART with the forward motor controller, which will be refereed to as \textit{the VESC} from this point on. The hardware necessary to control the forward motor using UART was implemented; the RX and TX wires had been connected between the MyRIO and the VESC, which in turn is connected to the forward motor. In LabVIEW there existed a VI which sent a command to the UART port connected to the VESC, however this program was undocumented and did not appear to work when it was tested.

Based on the previously created code, or lack there of, the LabVIEW code to communicate with the VESC using UART and consequently control the forward motor had to be created from scratch. This process with be described in the upcoming subsections. This process consist of configuring the VESC, encoding the correct UART command, successfully sending this command using LabVIEW, as well as some general testing and tuning.

\subsection{Configuring the VESC} \label{method:configuringVesc}

The development of the code meant to control the forward motor started with trying to use the already existing test program to send a UART command to the VESC. Even though the command appeared to match the description from \ref{theory:vesc}, the motor did not react. Why this was the case was unknown, which created doubts concerning which part of the setup caused the problem.

In an attempt to narrow down the cause of the problem, a computer running the VESC Tool was connected directly to the VESC via USB. Using this tool, an \textit{experiment} was made where a fixed RPM was sent to the controller. This resulted in a high pitched noise, as well as close to no movement of the motor.

It was hypothesized that the VESC was misconfigured in regard to the specifications of the forward motor. To try to fix this issue, the setup wizard built into the tool was used. In the wizard, the requested values and parameters was set to that of the motor \cite{Shimano2022DU-E6010}. The values which could not be found in the forward motor's data sheet were set to the recommended values. When running the same experiment as before and setting the motor's RPM to a fixed value, the motor now turned at what seemed to be the correct speed, without any screeching.

The VESC Tool is also used to make sure the VESC is set to use UART as its communication interface. This is done by changing the \textit{APP to Use} to UART under \textit{App Settings > General}. The baud rate can also now be accessed and changed, but is kept at the default value of 115 200 bps.

\subsection{Encoding UART command}

The UART communication with the VESC is based on sending various commands which should make the motor react in a certain way. Apart from the blog post by Vedder cited in section \ref{theory:vesc} as well as his example for the STM32F4 discovery board, there exists no documentation for how the UART commands sent to the VESC should look. 

The first attempt at encoding a UART command was made almost entirely in LabVIEW by setting a RPM value in the front panel, flipping its endianness, and wrapping it in start and stop bytes as well as checksum bytes. The payload was also split into separate bytes in the same way as in Vedder's example. Due to lacking documentation regarding the UART implementation, and there being no easy way of knowing exactly what bytes a correctly encoded message should contain, this method was later abandoned.

It was instead decided that a shared library would be created based on the example made by Vedder; the shared library could then be called using a \textit{Call Library Function Node} in LabVIEW. This method circumvented the problem of having to write the communication in a, for this purpose, undocumented language. Instead the correct command could be returned from a C function based on an inputted RPM value. 

To create this library which could be called by LabVIEW, the code that created the desired UART commands had to be identified. The functions to call and send specific commands was located in the \texttt{bldc\_interface.c} file. The three functions that have been adapted to work with the myRIO are \texttt{bldc\_interface\_set\_rpm}, \texttt{bldc\_interface\_set\_current}, as well as  \texttt{bldc\_interface\_send\_alive}. They have also been renamed to \texttt{setRpm}, \texttt{setCurrent}, and \texttt{sendAlive} respectively, to better represent  their new purpose. These functions was chosen since they were deemed to be the most useful commands related to controlling the forward motor. Functions later called by these functions have also been adapted and, in some cases, simplified. The functions \texttt{buffer\_append\_int32} and \texttt{buffer\_append\_float32} converts an integer or float values to a big-endian array, and \texttt{send\_packet\_no\_fwd} then calls a function to build and send the actual packet. In the adapted code, there is instead a function named \texttt{buildPacket} which builds and returns the packet. This function also calls \texttt{crc16} to calculate and add the checksum bytes. Finally the packet is copied to an array which is passed as a pointer to \texttt{setRPM} and \texttt{setCurrent}.

\subsection{Sending UART command using LabVIEW}

To create the UART command using LabVIEW, the desired RPM or current is sent to a \textit{Call Library Function Node} which is set to call the correct function from the C code, after it has been compiled and uploaded to the myRIO. The called function has an additional input parameter in the form of a pointer to an array. This array is initialized in LabVIEW, so that the pointer of it can be sent to the C code. When the C code runs, the pointer is used to populate the array with the values of the command which should be sent to the VESC. 

The actual transmission of the command is accomplished using the \textit{UART} block in LabVIEW. The block is configured to use the UART channel corresponding to the connection with the VESC, the baud rate is also set to 115 200 bps which is the same value as was previously configured in the VESC Tool. The transmission is placed inside of a loop to prevent the VESC from going to sleep.

The loop which which was desired to be used was a \textit{Timed Loop}, which executes at a precise rate specified by the user. Though due to problems with the program crashing, the type of loop was changed to a \textit{While Loop}; the cause of these crashes are unknown. A delay of one second was also introduced inside the loop to prevent it from throttling the rest of the system. The change of loop type did resolve the program crashing, though did also introduce another issue with recording loop times. This is further discussed in \ref{method:loopTimes}.

\section{Cleaning up and organizing the code}

In an attempt to make it easier for both the current and future people working on the Autobike project, all of the LabVIEW and C code that is used has been moved into a single folder. Previously it was separated into two different folders which had to be placed in specific locations on the development computer for LabVIEW to be able to detect them. The folder has also been uploaded to a monolithic public GitHub repository\footnote{https://github.com/Hannnes1/autobike}. The purpose of this is to make it easier to collaborate on the code, as well as making it possible to keep track of changes. It should also make it easier for others to take over development of the bike, as this was previously unnecessarily tedious. However, due to how LabVIEW functions, there is still one file which has to be manually located when opening LabVIEW on a new computer, this file is located inside \texttt{labview/bin} together with all other files that the LabVIEW project uses.

The LabVIEW code that was received from the previous project group has been cleaned up, meaning that the main program has been separated into several sub-VIs, and unused code has been removed. The main advantage of creating sub-VIs is to reduce the area which the LabVIEW code uses; a large area makes it hard to locate code since you can not zoom inside LabVIEW. The code that has been removed are mainly VIs that were either old versions of the main VI, or old code used for testing individual components (presumably from before this was integrated into the main VI). By removing unused VIs the potential confusion about what a VI does, and which VI is the most recent one, is mitigated.

Care has also been taken to document the new code that has been written, as well as the old code, so that anyone can continue the development process without having to interpret all of the code without any guidance. In addition to writing comments inline in the C and LabVIEW files, a \texttt{README.md} file has also been created in the root of the repository. This file lists all the folders that are currently present and explains their purpose. When relevant, this file also contains instructions for how the content of the folder should be used.

\section{Logging of Control and Sensor Signals} \label{method:logging}

With the purpose of being able to record and log the control and sensor signals of the bike, a logging system had to be created. This system would allow for future logs of a test being plotted and analyzed, helping the tuning process of the system. The logs could also be used to troubleshoot potential problems with the bike's performance.

In this specific case to be able to examine the data from performed tests, values from sensors, actuators, and the balancing algorithm added to the logging system. LabVIEW has built in functionality for this using the \textit{TDMS file format} \cite{NationalInstruments2022TheFormat}, which includes blocks for opening and creating, streaming to, and closing TDMS files. These files can then be opened using a number of different methods. Some of these methods include a special LabVIEW VI, Excel, or MATLAB using the \textit{Data Acquisition Toolbox}. 

In the main LabVIEW program used on the bike, a TDMS file is either created or replaced in the specified location using a \textit{TDMS Open} block. In each location where data should be logged, a \textit{TDMS Write} block is added and a group name and channel name is specified. Data is inputted as an array where each element corresponds to one channel (when the data layout has been specified as \textit{interleaved}). Each time that data is sent to the TDMS block, new data points are added to the channels. Therefore by placing the block inside of a loop, data from each iteration of the loop will be logged to the file. When a test is completed and all of the test data has been recorded, the file is closed using a \textit{TDMS Close} block before the program finishes.

MATLAB was chosen as the method for analyzing the data which the logging system gathers. MATLAB was specifically chosen due to the flexibility it grants regarding how the data can be presented. The code to do perform the plotting is presented in appendix \ref{code:plotting},

\section{Testing and Validating}

During the development of the bike's software, several tests have been done continuously to ensure that the developed features worked as expected. These tests include both simpler and informal tests designed to validate the bike's most basic functions, and more formal tests where values of different signals were logged. A test protocol was also constructed to be able to recreate the performed tests. It is the methodology behind these tests which will be presented in this section. The corresponding results are presented in the upcoming chapter \ref{results}.

\subsection{Testing the Basic Functionality of the Motors}

The first tests which were performed consisted of running the two motors at constant speeds. These tests where made to ensure that the motors worked as expected, without fluctuating speeds. 

For the steering motor the first test was performed by running the existing LabVIEW code and adjusting the duty cycle of the motor using the slider on the front panel. During this test the bike's front wheel was suspended midair so it could rotate freely. After the balancing algorithm had been created and connected to control the steering motor, it was tested by leaning the bike and examining if the steering motor rotated in the same direction. This purpose of the latter test was only to insure the steering motor reacted when the roll rate of the bike changed, and not that the calculated control signals were reasonable.

The first test of the forward motor was described in \ref{method:configuringVesc}. Summarized, it was performed by using the VESC Tool to run the motor at a fixed RPM. Later on, when the code to send UART commands via LabVIEW had been created, a fixed RPM was instead sent to the motor via one of these commands. Both of the tests mentioned above were performed when the bike was stationary and elevated into the air, this to insure the bike would not start moving.

\subsection{Testing the Balancing Algorithm}

After concluding that both motors worked separately, the balancing algorithm was able to be tested with both motors running. Two types of tests were performed, the first using a bike roller, and the second allowing the bike to move unaided. Both of these tests are described in the subsections below.

\subsubsection{Bike Roller Test}

The bike roller test made it possible to replicate the effects of going forwards while not requiring a large amount of space. The test consisted of placing the bike on the roller and turning on both motors. For safety the bike was held, though this was done whilst trying to not manually interfere in its movement. The aim of this test was to examine if the motors worked in tandem, and if it seemed as the bike could potentially balance on its own. It should be noted that a large restricting factor of this test is the limited space the bike can move sideways before reaching the boundary of the roller.

\subsubsection{Unaided Test} \label{method:unaidedTest}

To test the bike using an approach closer to a real world environment, a series of tests were performed outdoors in a flat and open area. This test made it possible to drive the bike forward, as well as allowing it to move sideways, mitigating the limit of the previous bike roller test. To prevent the bike from falling and potentially being damaged, two people ran alongside the bike ready to catch in the case it would fall.

For the purpose of being able to subsequently analyze the bike's behavior, the tests where both filmed and recorded using the logging system presented in \ref{method:logging}. Multiple tests were made with different values for the PID gains, to see how changing these values would affect the bikes performance. 

The data which was chosen to be recorded, logged and plotted with MATLAB include the position and speed of the steering motor, the duty cycle sent to the steering motor, as well as the roll rate of the bike. The results of some of these tests are shown in section \ref{results:balancing}.

\subsection{Testing the Program's Loop Times} \label{method:loopTimes}

While not being apart of the project's aim in itself, the performance of the program is of interest. This is because the bike's balancing could potentially be impacted if parts of the program takes too long to run. To be able to measure the performance, the time to run one iteration of each loop in the LabVIEW program was recorded. These times are then added together to get a total loop time for the program. The members of the previous project group have done a more in depth examination of this performance, which means that the values which are retrieved during the current testing can be compared with those values. It should be noted that because a while-loop had to be used for the forward motor control, the time it takes to run can not be measured. However, since a delay of one second runs in the loop, that is the loop time unless the other code in the loop takes longer to compute (which it does not). Additionally, the loop time of the forward motor control does not matter in this stage of the project, since the speed of the motor will be kept constant.

\subsection{Test Protocol}

To simplify the process of replicating the tests, a testing protocol was created. All of the tests which had been previously performed which also were seen as reasonable to replicate were added. Some suggestions for tests which could help calibrate the balancing algorithm for either the red bike or other future versions were also added. The testing protocol includes descriptions of the tests, the desired results and eventual comments. The protocol can be found in appendix \ref{testingProtocol}.