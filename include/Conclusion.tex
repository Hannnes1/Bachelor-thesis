% CREATED BY MAGNUS GUSTAVER, 2020
\chapter{Conclusion}

The main goal of the project was to develop the software for the Autobike so that it could balance on its own whilst driving forward. In practice this meant completing two sub-goals: implementing an algorithm which, using the roll rate of the bike, could calculate the correct duty cycle for the steering motor, as well as writing software to encode a UART command which then could be used to control the forward motor.

As described in the results, the two sub-goals as well as the primary aim of the project were reached. The forward motor can be controlled from LabVIEW, and the steering motor is controlled by the balancing algorithm. The balancing algorithm can also successfully keep the bike stable, at least until the forward velocity becomes too low. What this minimum velocity is has not been determined, and will vary depending on what the gains of the PID controller are. Even if the some pieces of hardware were not properly mounted, this did not affect the performance in any meaningful way in this stage of the development. If the controller were to be tuned further, the hardware should probably be securely mounted so that the best possible foundation is achieved.

The secondary goal of the project, improving the future development experience, is as previously discussed difficult to judge if it has been reached. The wording "improving" does not contain the degree of which this should be done, meaning any small improvement could be interpreted as reaching the goal. However, continuously during the project's development, files have been organized and cleaned up; both for the benefit of the current developers, but more importantly for any future developers. The question of whether this goal has been reached or not is therefore best judged by these future developers.

\section{Future work}

The priority for any future work should be to fix forward motor. Even if this is not something which affected the results of this report, the speed might become more important in the future if the current algorithms are improved, and additional ones are added. To accurately set the forward speed and to increase how far the bike can be driven, the forward motor would have to work.

Future work should also be focused on improving the hardware, and primarily its mounting. The IMU is the most important hardware component for the balancing algorithm, and will therefore have to be mounted so that its axis aligns with the axis of the bike. It also has to be mounted in such a way that it does not move around or vibrate excessively.